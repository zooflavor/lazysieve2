\documentclass{article}
\usepackage[magyar]{babel}
\usepackage[utf8]{inputenc}
\usepackage[T1]{fontenc}
\usepackage{amsfonts}
\usepackage{amssymb}
\usepackage{amsthm}
\usepackage{color}
\usepackage{mathtools}
\usepackage{algorithm}
\usepackage{algorithmicx}
\usepackage{algpseudocode}
\usepackage{listings}
\usepackage{multirow}
\usepackage{pgfplots}
\usepackage{tikz-qtree}

\lstset{literate=
  {á}{{\'a}}1 {é}{{\'e}}1 {í}{{\'i}}1 {ó}{{\'o}}1 {ú}{{\'u}}1
  {Á}{{\'A}}1 {É}{{\'E}}1 {Í}{{\'I}}1 {Ó}{{\'O}}1 {Ú}{{\'U}}1
  {à}{{\`a}}1 {è}{{\`e}}1 {ì}{{\`i}}1 {ò}{{\`o}}1 {ù}{{\`u}}1
  {À}{{\`A}}1 {È}{{\'E}}1 {Ì}{{\`I}}1 {Ò}{{\`O}}1 {Ù}{{\`U}}1
  {ä}{{\"a}}1 {ë}{{\"e}}1 {ï}{{\"i}}1 {ö}{{\"o}}1 {ü}{{\"u}}1
  {Ä}{{\"A}}1 {Ë}{{\"E}}1 {Ï}{{\"I}}1 {Ö}{{\"O}}1 {Ü}{{\"U}}1
  {â}{{\^a}}1 {ê}{{\^e}}1 {î}{{\^i}}1 {ô}{{\^o}}1 {û}{{\^u}}1
  {Â}{{\^A}}1 {Ê}{{\^E}}1 {Î}{{\^I}}1 {Ô}{{\^O}}1 {Û}{{\^U}}1
  {œ}{{\oe}}1 {Œ}{{\OE}}1 {æ}{{\ae}}1 {Æ}{{\AE}}1 {ß}{{\ss}}1
  {ű}{{\H{u}}}1 {Ű}{{\H{U}}}1 {ő}{{\H{o}}}1 {Ő}{{\H{O}}}1
  {ç}{{\c c}}1 {Ç}{{\c C}}1 {ø}{{\o}}1 {å}{{\r a}}1 {Å}{{\r A}}1
  {€}{{\euro}}1 {£}{{\pounds}}1 {«}{{\guillemotleft}}1
  {»}{{\guillemotright}}1 {ñ}{{\~n}}1 {Ñ}{{\~N}}1 {¿}{{?`}}1
}

\pgfplotsset{compat=1.9}

\begin{document}

\tableofcontents

\section{Bevezető}

Ezt a sievecompress  programot azért írtam, hogy legyen hol kipróbáljak ötleteket a szitatáblák tömörítésére.

Különösebb izgalom nincs benne. Fogtam egy marék elkészült szitatáblát, amit a szakdolgozattal generáltam. Vettem vad ötleteket, és megírtam a tömörítést. Ha nem volt reménytelen se a tömörítés ideje, se a tömörített méret, akkor megírtam a kitömörítést is ellenőrzésnek.

A tömörítés idejével nagyon nem törődtem, megelégedtem azzal, ha 10-20-30 perc alatt ki tudtam próbálni a marék szitatáblán, és el tudtam képzelni, hogy szükség esetén hatákonyan implementálható.

\section{Eredemény}

A vagdalkozásban találtam olyan algoritmust, ami a szitatábla egyszeri sorban végigolvasásával a Huffman-kódhoz hasonlóan jó aránnyal tömöríti a szitatáblát. A hasonlóan jó azt jelenti, hogy kb. +12-15\% a Huffman-hoz képest.

Van egy nagy táblázat az eredményről, az out.ods. Az oszlopai:
\begin{itemize}
\item segment start
\item ln(end): a szegmeny utolsó számának logaritmusa, a prímszámtétel szerint ekkora egy átlagos prímhézag
\item huffman/uncompressed
\item var1/uncompressed
\item var1 var0123: a var1 tömörítéshez használt paraméterek
\item log\_2(ln(end)/2): a var1 paraméterének tippje
\item var2/uncompressed
\item var2 var0123: a var2 tömörítéshez használt paraméterek
\item log\_2(ln(end)/6): a var2 paraméterének tippje
\item block(bitmap)/uncompressed
\item block(bitmap) length: a használt blockméret
\end{itemize}

\section{Algoritmusok}

Ezek elég pongyola leírásai az ötleteknek. Szükség esetére ott van a futtatható kód.

\subsection{Huffman}

\begin{itemize}
\item a szitatáblát bitmap-ből átalakítja a prímek listájává
\item a prímek listáját átalakítja a prímhézagok listájává
\item megszámolja a prímhézagok előfordulását
\item előállítja a kódfát
\item kódolja a prímhézagok listáját
\end{itemize}

Ez a legjobb tömörítés, amit kipróbáltam. Legalább kétszer végig kell olvassa a szitatáblát. A hatékony tömörítést el tudom képzelni szótárral, a kitömörítést nem annyira. Biztos lehetne erről sokat olvasni.

\subsection{Var1}

A "var" a variable length encoding-ból született.

\begin{itemize}
\item a szitatáblát bitmap-ből átalakítja a prímek listájává
\item a prímek listáját átalakítja a prímhézagok listájává
\item gondol 4 darab kicsi pozitív egészre, $a$, $b$, $c$, $d$
\item kódolja a prímhézagokat sorban:
\begin{itemize}
\item osztja kettővel a prímhézagot
\item a kimenetbe átshifteli a hézag alsó $a$ bitjét
\item ha a hézag 0 lett, akkor kiír egy 0 bitet
\item különben kiír egy 1 bitet, és kishiftel $b$ bitet a hézagból
\item ha a hézag 0 lett, akkor kiír egy 0 bitet
\item különben kiír egy 1 bitet, és kishiftel $c$ bitet a hézagból
\item ha a hézag 0 lett, akkor kiír egy 0 bitet
\item különben kiír egy 1 bitet, és kishiftel $d$ bitet a hézagból
\item amíg nem 0 a hézag
\begin{itemize}
\item kiír egy 1 bitet
\item kishiftel 1 bitet a hézagból
\end{itemize}
\item kiír egy 0 bitet
\end{itemize}
\end{itemize}

A számok kigondolását úgy írtam meg, hogy összeszedi a hézagok előfordulását, és brute-forca-szal megkeresi azt, ami a legkisebb méretet eredményezné.

A kis tesztem alapján a következő jött ki:
\begin{itemize}
\item csak az első szám külöbözik 1-től
\item az első szám is olyan, mintha nagyon lassan növekedő monoton valami lenne
\end{itemize}

Az egymenetben tömörítést úgy képzelem, hogy ezt az egyetlen lassúcska paramétert előzetes tesztek alapján beégetjük. A hatékonyságot meg az adhatja, hogy az egész csak shift-ekből és konstansokkal összehasonlításokból áll.

\subsection{Var2}

Szinte ugyanaz, mint a var1. A különbség az, hogy a prímhézagot nem kettővel osztja, hanem
\begin{itemize}
\item ha a hézag mod 6 = 0, akkor kiír egy 0 bitet
\item ha a hézag mod 6 = 2, akkor kiírja az 10 biteket
\item ha a hézag mod 6 = 4, akkor kiírja az 11 biteket
\item osztja hattal a hézagot
\end{itemize}

A tesztek alapján úgy tűnik, hogy elég messze szitálva egy icipicivel jobb, mint a var1. Nehéz megmondani, hogy megéri-e az extra komplexitást.

\subsection{Block(bitmap)}

Az inverz szita alapján gondoltam, hogy próbálkozok olyan tömörítéssel, ami a szitatábla bitmap-jét blockokra vágva tömörít. Nem sok sikert értem el, de legalább minden ötletem nagyon lassú is volt.

Egyet azért megtartottam, hogy újabb ötletnél legyen valami kiindulásnak.

\begin{itemize}
\item gondol egy blokkméretre
\item veszi sorban a blokkokat
\begin{itemize}
\item ha egy prím sincs a blokkban, akkor kiír egy 0 bitet
\item különben kiír egy 1 bitet, és a blokk bitmapjét
\end{itemize}
\end{itemize}

\end{document}
